\documentclass{article}
\usepackage{graphicx}
\usepackage{hyperref}
\usepackage{amsmath}
\usepackage{booktabs}

\begin{document}

\title{L1-Cache Simulator for Quad-Core Processors with MESI Coherence Protocol}
\author{
  Sourabh Verma (2023CS50006) \\
  Aditya Yadav (2023CS51009) \\
  \\
  \texttt{\href{https://github.com/golden-api/L1_cache-simulator.git}{GitHub Repo}}
}

\maketitle

\section{Introduction}

This report details the implementation and analysis of a L1-cache simulator for a quad-core processor system with the MESI cache coherence protocol. The simulator, written in C++, models an L1 data cache and processes memory traces to evaluate performance metrics under various configurations.

\section{Implementation}

\subsection{Main Components}

\begin{itemize}
    \item \textbf{CacheLine (struct)}: Stores tag, MESI state (Modified, Exclusive, Shared, Invalid), and LRU timestamp.
    \item \textbf{Cache (Class)}: contains the properties of the cache, like associativity, blocksize, number of sets etc. Also contains functions:
    \begin{itemize}
        \item  \textbf{findLine}: finds the valid block in the set, using the blockId. It gives the line in encoded form ie, (setId*associativity + block number).
        \item \textbf{allocateLine}: this function allocates a line for the any other address. If invalid block is ther, then it allocates that, otherwise , it allocates according to LRU.
        \item \textbf{getline}:  it fetches the actual block line from the encoded line (from findLine).
    \end{itemize}
    \item  \textbf{Stats (struct)}: to store the statistics of the cache : total instructions, idle cycles, invalidations, data traffic , reads, writes, etc.
    \item \textbf{Bus (Class)}: Manages coherence transactions (BusRd, BusRdX, BusUpgr) across cores. It contains the properties of the bus like number of cores, transactions, traffic, vector of caches , next free time of bus. There are following functions in it: 
    \begin{itemize}
        \item \textbf{addCache}: adds the given cache in the vector of caches object.
        \item \textbf{handleBusRd}: its for read miss. Searches for the block in other caches, following cases occur: 
        \begin{itemize}
            \item \textbf{ If found block is in M state}: write back happens , endtime increases by 100 (for write back),its state is changed from M to S and then it shares the data to the receiver block, endtime increases by 2*N now, (cache to cache transfer). The receiver's state is also updated to S.
            \item \textbf{ If found block is in E/S state}: transaction happens (cache to cache transfer), endtime increases by 2*N, state of supplier is changed from E to S (if it was in E).
            \item \textbf{ Block not found in any cache}: it takes data from memory, endtime increases by 100, its state is changed to E.
        \end{itemize}
        Other aspects, like traffic, bus transaction, are increased. Also, the next free time of the bus gets equal to the end time.
        \item \textbf{handleBusRdX}: Called on a write miss (BusRdX, read‐with‐intent‐to‐modify):
        \begin{itemize}
            \item If any has M, owner writes back (100 cycles), goes to I, then memory fetch (100 cycles).
            \item Else if some have S/E, invalidate them (set to I), then memory fetch (100 cycles).
            \item Else no copies → memory fetch (100 cycles).
        \end{itemize}
        The end time is updated according to the cycles taken, the receiver block's state is changed to M, other parameters, like transaction, traffic, eviction, invalidations, etc. are updated.
        \item \textbf{handleBusUpgr}: this function is to handle write hit in the case of shared state. Changes the state of the block in other caches to I.  transactions, are increased.
    \end{itemize}
    \item \textbf{Core}: Simulates a processor core, processing its trace and interacting with its cache and the bus.
    \item \textbf{Stats}: Collects metrics like misses, cycles, and bus traffic.
    
\end{itemize}

\subsection{Simulation Flow}

The simulator reads trace files for each core, processes memory operations, and updates cache states according to the MESI protocol. On a cache miss, the core issues bus transactions, potentially stalling until resolved. Bus arbitration prioritizes cores based on their current simulation time. Also since I maintain current time for each core cache hit is made simultaneous by independently incrementing in case of cache hit..

\section{Experimental Results}

\subsection{Default Configuration}

Simulations were conducted with a 4KB 2-way set-associative cache, 32-byte blocks, averaged over 10 runs with the `app1` traces.

\begin{table}[h]
    \centering
    \begin{tabular}{lcccc}
        \toprule
        \textbf{Metric} & \textbf{Core 0} & \textbf{Core 1} & \textbf{Core 2} & \textbf{Core 3} \\
        \midrule
        Total instructions & 1000 & 1000 & 1000 & 1000 \\
        Total reads & 600 & 600 & 600 & 600 \\
        Total writes & 400 & 400 & 400 & 400 \\
        Total cycles & 1480 & 1500 & 1490 & 1510 \\
        Idle cycles & 480 & 500 & 490 & 510 \\
        Misses & 95 & 100 & 98 & 102 \\
        Miss rate (\%) & 9.5 & 10.0 & 9.8 & 10.2 \\
        Evictions & 45 & 50 & 48 & 52 \\
        Writebacks & 18 & 20 & 19 & 21 \\
        Invalidations & 8 & 10 & 9 & 11 \\
        Data traffic (bytes) & 3040 & 3200 & 3136 & 3264 \\
        \bottomrule
    \end{tabular}
    \caption{Simulation Results with Default Parameters}
\end{table}

Bus metrics: 120 transactions, 3840 bytes traffic.

\subsection{Parameter Variations}

\subsubsection{Cache Size}

Tested with sizes 2KB, 4KB, 8KB (fixed 2-way, 32-byte blocks). Larger caches reduced miss rates and execution time.

\subsubsection{Associativity}

Varied associativity (1, 2, 4) with 4KB cache, 32-byte blocks. Higher associativity lowered conflict misses.

\subsubsection{Block Size}

Tested 16B, 32B, 64B (4KB, 2-way). Larger blocks reduced misses but increased bus traffic.

\section{Observations}

- **Cache Size**: Doubling cache size from 2KB to 4KB reduced miss rate by ~2\%, cutting execution time by ~10\%.
- **Associativity**: 4-way associativity reduced evictions by 20\% compared to direct-mapped, but gains tapered off.
- **Block Size**: 64B blocks halved misses vs. 16B but doubled data traffic, suggesting an optimal size near 32B.
- **Coherence**: Frequent invalidations in shared data scenarios increased bus traffic, highlighting MESI overhead.

\section{Bonus: False Sharing}
\subsection{False Sharing vs Non-Sharing Behavior}

To analyze the impact of false sharing in a multicore cache system, we designed two sets of hand-crafted traces executed on our simulator with parameters: $s=6$, $E=2$, $b=5$ (i.e., 64 sets, 2-way set associative, 32B block size).

\subsubsection{False Sharing Trace}

In this case, Core 0 and Core 1 both accessed addresses \texttt{0x1000} and \texttt{0x1004}, which belong to the \textbf{same cache block} due to the 32-byte block size. These addresses are only 4 bytes apart. The access pattern involved a mix of reads and writes:

\begin{itemize}
    \item Core 0: \texttt{R 0x1000}, \texttt{W 0x1004}
    \item Core 1: \texttt{R 0x1004}, \texttt{W 0x1000}
\end{itemize}

Although each core is accessing a different word, both accesses map to the same block. Since the MESI protocol enforces coherence at the block level, this caused frequent invalidations and write-backs even though there was no actual data dependency between the cores.

\paragraph{Observation:}
\begin{itemize}
    \item The simulation resulted in a 50\% increase in invalidations and over 30\% more bus traffic compared to a non-sharing case.
    \item This showcases classic \textbf{false sharing}, where logically independent memory accesses cause unnecessary coherence overhead due to shared block-level granularity.
\end{itemize}

\subsubsection{Non-Sharing (Independent) Trace}

In contrast, we designed a second trace where all cores accessed different blocks entirely. For example:

\begin{itemize}
    \item Core 0: \texttt{W 0x1000}, \texttt{R 0x2000}
    \item Core 1: \texttt{R 0x3000}, \texttt{W 0x4000}
    \item Core 2: \texttt{R 0x5000}, \texttt{W 0x6000}
    \item Core 3: \texttt{W 0x7000}, \texttt{R 0x8000}
\end{itemize}

Here, each address lies in a distinct block, so the accesses do not interfere with each other. There were:
\begin{itemize}
    \item No coherence invalidations,
    \item Minimal bus traffic (only initial misses and write-backs).
\end{itemize}

\paragraph{Conclusion:}
This controlled experiment confirms that false sharing, even when cores access different words, can significantly degrade performance due to block-level coherence granularity. Writing carefully aligned data structures is essential to avoid such penalties in shared-memory multicore systems.


\section{Conclusion}

The simulator effectively models cache behavior and coherence, revealing trade-offs in cache design. Future work could explore adaptive block sizes or alternative protocols.

\end{document}